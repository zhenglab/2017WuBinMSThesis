\chapter{数据库备份脚本}
\label{cha:MysqlBackup}
\begin{lstlisting}[numbers=none]
#!/bin/bash
#数据库用户名
dbuser='root'
#数据库用密码
dbpasswd='password'
#需要备份的数据库,多个数据库用空格分开
dbname='minishop'
#备份时间
backtime= (date +%Y%m%d%H%M%S)
#日志备份路径
logpath='/mnt/mysqldump'
#数据备份路径
datapath='/mnt/mysqldump'
#日志记录头部
echo "备份时间为${backtime},备份数据库表 ${dbname} 开始" >> ${logpath}/log.log
#正式备份数据库
for dbn in $dbname; do
source=(mysqldump -u${dbuser} -p${dbpasswd} -hdbip -Pdbport ${dbn} --set-gtid-purged=OFF > ${logpath}/${backtime}.sql) 2>> ${logpath}/mysqllog.log;
#备份成功以下操作
if [ "$?" == 0 ];then
cd $datapath
#为节约硬盘空间,将数据库压缩
tar jcf ${dbn}${backtime}.tar.bz2 ${backtime}.sql > /dev/null
#删除原始文件,只留压缩后文件
rm -f ${datapath}/${backtime}.sql
#上传到oss
/mnt/sh/oss/dbuposs.py ${dbn}${backtime}.tar.bz2
#删除七天前备份,也就是只保存7天内的备份
find $datapath -name "*.tar.bz2" -type f -mtime +7 -exec rm -rf {} \; > /dev/null 2>&1
echo "数据库表 ${dbname} 备份成功!!" >> ${logpath}/mysqllog.log
else
#备份失败则进行以下操作
echo "数据库表 ${dbname} 备份失败!!" >> ${logpath}/mysqllog.log
fi
done
\end{lstlisting}