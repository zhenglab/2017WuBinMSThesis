\thusetup{
  %******************************
  % 注意:
  %   1. 配置里面不要出现空行
  %   2. 不需要的配置信息可以删除
  %******************************
  %
  % 中国海洋大学研究生学位论文封面
  % 参考:中国海洋大学研究生学位论文书写格式20130307.doc
  % 为避免出现错误,下面保留[清华大学学位论文模板原有定义无需修改],
  % 请直接跳到后面[中国海洋大学学位论文模板部分请根据自己情况修改]。
  %
%%%%%%%%%%%%%%%%%%%%%%[清华大学学位论文模板原有定义无需修改]%%%%%%%%%%%%%%%%%%%%%%%
  %=====
  % 秘级
  %=====
  secretlevel={秘密},
  secretyear={10},
  %
  %=========
  % 中文信息
  %=========
  ctitle={清华大学学位论文 \LaTeX\ 模板\\使用示例文档 v\version},
  cdegree={工学硕士},
  cdepartment={计算机科学与技术系},
  cmajor={计算机科学与技术},
  cauthor={薛瑞尼},
  csupervisor={郑纬民教授},
  cassosupervisor={陈文光教授}, % 副指导老师
  ccosupervisor={某某某教授}, % 联合指导老师
  % 日期自动使用当前时间,若需指定按如下方式修改:
  % cdate={超新星纪元},
  %
  % 博士后专有部分
  cfirstdiscipline={计算机科学与技术},
  cseconddiscipline={系统结构},
  postdoctordate={2009年7月——2011年7月},
  id={编号}, % 可以留空: id={},
  udc={UDC}, % 可以留空
  catalognumber={分类号}, % 可以留空
  %
  %=========
  % 英文信息
  %=========
  etitle={An Introduction to \LaTeX{} Thesis Template of Tsinghua University v\version},
  % 这块比较复杂,需要分情况讨论:
  % 1. 学术型硕士
  %    edegree:必须为Master of Arts或Master of Science(注意大小写)
  %             “哲学、文学、历史学、法学、教育学、艺术学门类,公共管理学科
  %              填写Master of Arts,其它填写Master of Science”
  %    emajor:“获得一级学科授权的学科填写一级学科名称,其它填写二级学科名称”
  % 2. 专业型硕士
  %    edegree:“填写专业学位英文名称全称”
  %    emajor:“工程硕士填写工程领域,其它专业学位不填写此项”
  % 3. 学术型博士
  %    edegree:Doctor of Philosophy(注意大小写)
  %    emajor:“获得一级学科授权的学科填写一级学科名称,其它填写二级学科名称”
  % 4. 专业型博士
  %    edegree:“填写专业学位英文名称全称”
  %    emajor:不填写此项
  edegree={Doctor of Engineering},
  emajor={Computer Science and Technology},
  eauthor={Xue Ruini},
  esupervisor={Professor Zheng Weimin},
  eassosupervisor={Chen Wenguang},
  % 日期自动生成,若需指定按如下方式修改:
  % edate={December, 2005}
  %
  % 关键词用“英文逗号”分割
  ckeywords={\TeX, \LaTeX, CJK, 模板, 论文},
  ekeywords={\TeX, \LaTeX, CJK, template, thesis}
}

% 定义中英文摘要和关键字
\begin{cabstract}
  论文的摘要是对论文研究内容和成果的高度概括。摘要应对论文所研究的问题及其研究目
  的进行描述,对研究方法和过程进行简单介绍,对研究成果和所得结论进行概括。摘要应
  具有独立性和自明性,其内容应包含与论文全文同等量的主要信息。使读者即使不阅读全
  文,通过摘要就能了解论文的总体内容和主要成果。

  论文摘要的书写应力求精确、简明。切忌写成对论文书写内容进行提要的形式,尤其要避
  免“第 1 章……;第 2 章……;……”这种或类似的陈述方式。

  本文介绍清华大学论文模板 \thuthesis{} 的使用方法。本模板符合学校的本科、硕士、
  博士论文格式要求。

  本文的创新点主要有:
  \begin{itemize}
    \item 用例子来解释模板的使用方法;
    \item 用废话来填充无关紧要的部分;
    \item 一边学习摸索一边编写新代码。
  \end{itemize}

  关键词是为了文献标引工作、用以表示全文主要内容信息的单词或术语。关键词不超过 5
  个,每个关键词中间用分号分隔。(模板作者注:关键词分隔符不用考虑,模板会自动处
  理。英文关键词同理。)
\end{cabstract}

% 如果习惯关键字跟在摘要文字后面,可以用直接命令来设置,如下:
% \ckeywords{\TeX, \LaTeX, CJK, 模板, 论文}

\begin{eabstract}
   An abstract of a dissertation is a summary and extraction of research work
   and contributions. Included in an abstract should be description of research
   topic and research objective, brief introduction to methodology and research
   process, and summarization of conclusion and contributions of the
   research. An abstract should be characterized by independence and clarity and
   carry identical information with the dissertation. It should be such that the
   general idea and major contributions of the dissertation are conveyed without
   reading the dissertation.

   An abstract should be concise and to the point. It is a misunderstanding to
   make an abstract an outline of the dissertation and words ``the first
   chapter'', ``the second chapter'' and the like should be avoided in the
   abstract.

   Key words are terms used in a dissertation for indexing, reflecting core
   information of the dissertation. An abstract may contain a maximum of 5 key
   words, with semi-colons used in between to separate one another.
\end{eabstract}

% \ekeywords{\TeX, \LaTeX, CJK, template, thesis}
%%%%%%%%%%%%%%%%%%%%%%%%%%%%%%%%%%%%%%%%%%%%%%%%%%%%%%%%%%%%%%%%%%%%%%%%%%%%%%%%

%%%%%%%%%%%%%%%%%%[中国海洋大学学位论文模板部分请根据自己情况修改]%%%%%%%%%%%%%%%%%%%
% 中国海洋大学研究生学位论文封面
% 必须填写的内容包括(其他最好不要修改):
%   分类号、密级、UDC
%   论文中文题目、作者中文姓名
%   论文答辩时间
%   封面感谢语
%   论文英文题目
%   中文摘要、中文关键词
%   英文摘要、英文关键词
%
%%%%%[自定义]%%%%%
\newcommand{\fenleihao}{}%分类号
\newcommand{\miji}{}%密级
                    % 绝密$\bigstar$20年
                    % 机密$\bigstar$10年
                    % 秘密$\bigstar$5年
\newcommand{\UDC}{}%UDC
\newcommand{\oucctitle}{基于Spring MVC架构WEB应用系统优化研究}%论文中文题目
\ctitle{基于Spring MVC架构WEB应用系统优化研究}%必须修改因为页眉中用到
\cauthor{XX}%可以选择修改因为仅在 pdf 文档信息中用到
\cdegree{工学硕士}%可以选择修改因为仅在 pdf 文档信息中用到
\ckeywords{\TeX, \LaTeX, CJK, 模板, 论文}%可以选择修改因为仅在 pdf 文档信息中用到
\newcommand{\ouccauthor}{武斌}%作者中文姓名
%\newcommand{\ouccsupervisor}{姬光荣教授}%作者导师中文姓名
%\newcommand{\ouccdegree}{博\hspace{1em}士}%作者申请学位级别
%\newcommand{\ouccmajor}{海洋信息探测与处理}%作者专业名称
%\newcommand{\ouccdateday}{\CJKdigits{\the\year}年\CJKnumber{\the\month}月\CJKnumber{\the\day}日}
%\newcommand{\ouccdate}{\CJKdigits{\the\year}年\CJKnumber{\the\month}月}
\newcommand{\oucdatedefense}{2017年05月25日}%论文答辩时间
%\newcommand{\oucdatedegree}{2009年6月}%学位授予时间
\newcommand{\oucgratitude}{谨以此论文献给我的导师和亲人!}%封面感谢语
\newcommand{\oucetitle}{Research on Web Platform System Optimization Strategy Based on Spring MVC Framework}%论文英文题目
%\newcommand{\ouceauthor}{Haiyong Zheng}%作者英文姓名
\newcommand{\oucthesis}{\textsc{OUCThesis}}
%%%%%默认自定义命令%%%%%
% 空下划线定义
\newcommand{\oucblankunderline}[1]{\rule[-2pt]{#1}{.7pt}}
\newcommand{\oucunderline}[2]{\underline{\hskip #1 #2 \hskip#1}}

% 论文封面第一页
%%不需要改动%%
\vspace*{5cm}
{\xiaoer\heiti\oucgratitude

\begin{flushright}
---\hspace*{-2mm}---\hspace*{-2mm}---\hspace*{-2mm}---\hspace*{-2mm}---\hspace*{-2mm}---\hspace*{-2mm}---\hspace*{-2mm}---\hspace*{-2mm}---\hspace*{-2mm}---~\ouccauthor
\end{flushright}
}

\newpage

% 论文封面第二页
%%不需要改动%%
\vspace*{1cm}
\begin{center}
  {\xiaoer\heiti\oucctitle}
\end{center}
\vspace{10.7cm}
{\normalsize\songti
\begin{flushright}
{\renewcommand{\arraystretch}{1.3}
  \begin{tabular}{r@{}l}
    学位论文答辩日期:~ & \oucunderline{1.8em}{\oucdatedefense} \\
    指导教师签字:~ & \oucblankunderline{5cm} \\
    答辩委员会成员签字:~ & \oucblankunderline{5cm} \\
    ~ & \oucblankunderline{5cm} \\
    ~ & \oucblankunderline{5cm} \\
    ~ & \oucblankunderline{5cm} \\
    ~ & \oucblankunderline{5cm} \\
    ~ & \oucblankunderline{5cm} \\
    ~ & \oucblankunderline{5cm} \\
  \end{tabular}
}
\end{flushright}
}

\newpage

% 论文封面第三页
%%不需要改动%%
\vspace*{1cm}
\begin{center}
  {\xiaosan\heiti 独\hspace{1em}创\hspace{1em}声\hspace{1em}明}
\end{center}
\par{\normalsize\songti\parindent2em
本人声明所呈交的学位论文是本人在导师指导下进行的研究工作及取得的研究成果。据我所知,除了文中特别加以标注和致谢的地方外,论文中不包含其他人已经发表或撰写过的研究成果,也不包含未获得~\oucblankunderline{7cm}(注:如没有其他需要特别声明的,本栏可空)或其他教育机构的学位或证书使用过的材料。与我一同工作的同志对本研究所做的任何贡献均已在论文中作了明确的说明并表示谢意。
}
\vskip1.5cm
\begin{flushright}{\normalsize\songti
  学位论文作者签名:\hskip2cm 签字日期:\hskip1cm 年 \hskip0.7cm 月\hskip0.7cm 日}
\end{flushright}
\vskip.5cm
{\setlength{\unitlength}{0.1\textwidth}
  \begin{picture}(10, 0.1)
    \multiput(0,0)(0.2, 0){50}{\rule{0.15\unitlength}{.5pt}}
  \end{picture}}
\vskip1cm
\begin{center}
  {\xiaosan\heiti 学位论文版权使用授权书}
\end{center}
\par{\normalsize\songti\parindent2em
本学位论文作者完全了解学校有关保留、使用学位论文的规定,并同意以下事项:
\begin{enumerate}
\item 学校有权保留并向国家有关部门或机构送交论文的复印件和磁盘,允许论文被查阅和借阅。
\item 学校可以将学位论文的全部或部分内容编入有关数据库进行检索,可以采用影印、缩印或扫描等复制手段保存、汇编学位论文。同时授权清华大学“中国学术期刊(光盘版)电子杂志社”用于出版和编入CNKI《中国知识资源总库》,授权中国科学技术信息研究所将本学位论文收录到《中国学位论文全文数据库》。
\end{enumerate}
(保密的学位论文在解密后适用本授权书)
}
\vskip1.5cm
{\parindent0pt\normalsize\songti
学位论文作者签名:\hskip4.2cm\relax%
导师签字:\relax\hspace*{1.2cm}\\
签字日期:\hskip1cm 年\hskip0.7cm 月\hskip0.7cm 日\relax\hfill%
签字日期:\hskip1cm 年\hskip0.7cm 月\hskip0.7cm 日\relax\hspace*{1.2cm}}

\newpage

\pagestyle{plain}
\clearpage\pagenumbering{roman}

% 中文摘要
%%[需要填写:中文摘要、中文关键词]%%
\begin{center}
  {\sanhao[1.5]\heiti\oucctitle\\\vskip7pt 摘\hspace{1em}要}
\end{center}
{\normalsize\songti

  \indent
  近几年,随着国家经济的不断发展以及互联网整体水平的不断提高,虚拟
  经济在国家的经济发展中扮演着越来越重要的角色,WEB应用作为虚拟经济的
  主要体现方式发展迅猛。许多面向各级消费群体的WEB应用
  被设计和开发出来为他们提供便捷的互联网体验和服务。

  随着Internet的不断高速发展,人们对于WEB应用的需求不再基于可用性那
  么简单,应用的用户体验越来越被重视~\cite{李梦2016基于前端的}。然而,
  随着用户数量的不断增加,WEB服务器的信息量和访问量成几何数级的增长,网
  络拥塞和服务超载日益成为开发人员必须面对的严峻问题。

  因此在WEB应用的开发过程中,开发人员不但需要在代码的书写方面要进行不断的优化,
  降低代码的错误率同时提升代码的执行效率,而且需要将很大的精力投入到服务器性能
  优化方面,通过调整WEB服务器的运行参数提升服务器的响应效率,优化缓存机制
  来提升数据的存取效率,设计负载均衡方案来实现应用服务器的负载均衡,开发数据库
  的主从复制和延时复制等策略来保证数据的稳定性以及编写监控脚本来实现服务的实时
  监控等措施来提升应用的体验并增强系统的安全性。

  本文的创新点主要有:
  \begin{itemize}
    \item 通过Jenkins实现WEB应用的自动构建和部署,提升了WEB应用的稳定性,降低代码错误率;
    \item 基于Couchbase缓存机制,加快了应用的数据读取,降低了系统的响应时间;
    \item 通过Docker容器编排技术,方便的实现服务集群节点的扩展,增强了系统的安全性,实现了统一管理和维护;
    \item 通过负载均衡,降低了应用服务器和数据服务器的压力,提升了用户的体验;
    \item 充分利用API实现监控脚本,实现了应用的实时监控,很大程度上减少了发现问题和故障处理的时间;
  \end{itemize}

}
\vskip12bp
{\xiaosi\heiti\noindent
关键词:\hskip1em Spring MVC; WEB优化; 数据库优化; 服务监控; }

\newpage

% 英文摘要
%%[需要填写:英文摘要、英文关键词]%%
\begin{center}
  {\sanhao[1.5]\heiti\oucetitle\\\vskip7pt Abstract}
\end{center}
{\normalsize\songti

   In recent years, with the development of Chinese economy and Internet, virtual economy plays an increasingly important role in Chinese economic development. As the main presentation of virtual economy, WEB application gets very big development. More and more WEB applications are designed and developed to provide convenient Internet experience and service for consumers.

   Also, with the continuous development of the Internet, we need to know that our demand for WEB applications is no longer just based on usability, people also pay more attention to the user experience of WEB application. However, with the increase of the number of users, the amount of information and the number of visits of the WEB server grows geometrically. Network congestion and service overload are increasingly becoming the serious problem that the developers must face.

   Therefore, when we developing WEB applications, we not only need to optimize the code structure such as reduce the code error rate and improve the efficiency of the code, but also to need to pay more attention to optimize the server performance, such as adjusting the operating parameters of the WEB server to improve the response efficiency of the server, optimizing the cache mechanism to improve the efficiency of data access, designing load balancing solution to achieve the application server load balancing, design database replication strategy and delaying replication strategy to ensure data stability and design the monitoring scripts to achieve the implementation of service monitoring. The main purpose of this paper is design different type of measures to enhance the application experience and the security of the system.

   The innovation of this paper:
  \begin{itemize}
    \item Through Jenkins to realize the automatic construction and deployment of WEB applications to enhance the stability and reduce code error rate;
    \item Through Couchbase cache mechanism to speed up the data read of application, reducing the system response time;
    \item Through Docker container scheduling technology to facilitate the expansion of service cluster nodes to enhance the security of the system to achieve an unified management and maintenance;
    \item Reduces stress on application servers and data servers through load balancing to improve user experience;
    \item Making full use of API to design the monitoring script to achieve the real-time monitoring of application, so that it could reduce the time of sovling problems and troubleshooting;
  \end{itemize}
}
\vskip12bp
{\xiaosi\heiti\noindent
\textbf{Keywords:\enskip Spring MVC; WEB Optimization; Database Optimization; Service Monitoring }}
%%%%%%%%%%%%%%%%%%%%%%%%%%%%%%%%%%%%%%%%%%%%%%%%%%%%%%%%%%%%%%%%%%%%%%%%%%%%%%%%